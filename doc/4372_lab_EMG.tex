\chapter{Electromyography}

\section{Biopotentials}

Biopotentials are formed by ion concentration differences inside and outside a cell.  Membranes and specialized pumps in the membrane regulate and adjust the concentrations in response to external and internal stimulation, permitting the generation and propagation of biosignals.  Measuring the electrical potential in muscles is called electromyography or EMG.


Generally, doctors use needle electrodes, so the skin only needs to be wiped with alcohol to prevent infection, but we will not be using needle electrodes for safety and legal reasons (Texas Court of Appeals, Third District, at Austin, Cause No. 03-10-673-CV. April 5, 2012)++.

\section{Setup}

First, we will get the Arduino IDE.  You  we need to make sure we have the latest version of our software.   Open a terminal window and enter the following commands:

\CommandLine{sudo apt-get update}

\CommandLine{sudo apt-get upgrade}

\CommandLine{sudo apt-get install arduino}

The ArduBerry is a shield that allows Raspbery Pi's to use Arduino shields.  Type the following commands:

\CommandLine{cd code}

\CommandLine{git clone https://github.com/DexterInd/ArduBerry.git}

Open a web browser and go to \url{http://www.dexterindustries.com/Ardubecrry/getting-started/} and follow the instructions.  You will download and install the drivers and run a simple test script that verifies everything is working.  The brief summary is below (the website has pretty pictures to accompany this):

\begin{enumerate}
\item Stack ArduBerry on Raspberry Pi.
\item Go to the scripts directory in the ArduBerry repo :\CommandLine{cd ArduBerry/script}
\item Make the install script executable:\CommandLine{sudo chmod +x install.sh} then run it as root: \CommandLine{sudo ./install.sh} and follow prompts, pressing \CommandLine{enter} and \CommandLine{y} as needed.
\item The system should automatically reboot.
\item Open the Arduino IDE from the system menu.
\item From the \textbf{Tools} menu, select the \textbf{Programmer} sub-menu, then select \textbf{RaspberryPi GPIO}.
\item Load \textbf{blink} or another sample sketch, and press \textbf{CTRL}$+$\textbf{Shift}$+$\textbf{U} to upload (or select \textbf{Upload using Programmer} from the File menu)
\item Verify that the LED blinks (if running blink)
\end{enumerate}

\subsection{Olimex EKG/EMG Shield}

You should not have to perform any installs for this.  The shield is static sensitive so be careful, and the pins are often bent so be more careful.  Disconnect power.  Insert onto the Arduberry. Connect power.  From Pi launch Ardino editor, and set the target device to an uno and the programmer to GPIO. \textbf{Load arduberry emg sketch}, then \textbf{program}.

Now open a command window and cd to your EMG directory.  Run EMG.py.  This is a simple test program, that should give you 8 numbers (1-4 interspersed with something around 300).  If you get this all is well.

Now connect a volunteer to the ekg leads.  L goes on the left arm, R goes on the right arm, and D goes on the right leg.  They need to be symmetrically placed, i.e. all at wrist/ankle or elbow/knee, etc.

Run either the fixed (takes 2k samples then plots and holds) or plot (for dynamic plots).

\section{General Advice}

Gel electrodes provide a good measure but a few basic precautions should be followed to get the best signals.

\begin{itemize}
  \item Remove oils from your skin with soap and water to improve the signal.
  \item Don't apply lotions or creams.
  \item Make sure you are hydrated (dry skin doesn't conduct as well).
  \item Try to use smooth skin with minimal hair (callouses and hair make it more difficult to get a good signal).
  \item Body fat reduces the signal, so try to find areas where the muscle is as close to the surface as possible.
\end{itemize} 



\section{Test}

You will first need to setup the arduino on the shield to respond to our code.  From the application menu, pick the first menu group and the arduino ide should be the first pick.  In it, open the arduino sketch in our lab\_02 directory.  From the file menu select upload via programmer.  You are set.

Now open a terminal window and type

\CommandLine{cd c*/B*/l*/lab\_02}

\CommandLine{sudo python EMG\_plot.py}

After a second or two the plot window will open and start displaying the plot of the difference of R and L with D used as ground.  You can stop the plot with ctrl-c, though this will exit.  If you want to take a fixed data length and have the plot stay up then use

\CommandLine{sudo python EMG\_fixed.py}

I suggest using plot to try most of the experiments below.

Start with the metal plate electrodes.  Place ground on bicep.  Place the other two on the muscle the forearm that moves the wrist - one in the middle and one near the end.  What happens if you use the right leg?  Try metal plates and gel electrodes (just do arm)?

Place other two on the same forearm, one in the middle of a muscle and one near the end of the same muscle.  Locat muscles for two different fingers on the same hand by extending and contracting individually and noticing muscle flexure (i.e. I want you to do this twice).  What happens if you use the opposite arm  ?

Record signals for motion and identify finger flexed by chart only.  Explain.

How does the amount of exerted force relate to the signal?